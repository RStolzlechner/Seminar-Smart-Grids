\documentclass{article}
\usepackage{graphicx} % Required for inserting images

\title{Entwicklung eines Dashboards zur Darstellung von zeitbasierten Smart Grid Daten}
\author{ricardo.stolzlechner }
\date{April 2023}

\begin{document}

\maketitle

\section{Abstract}
Abstract ist letzte Text, den man schreibt!

Und so können Sie einen Abstract aufbauen (je Punkt ein halber bis maximal zwei Sätze):

\begin{enumerate}
    \item Beschreiben Sie die Ausgangssituation
    \item Wie ist der aktuelle Stand der Forschung?
    \item Welchen Forschungsbedarf haben Sie erkannt?
    \item Welche Forschungsfrage leitet sich ab?
    \item Was wurde wie untersucht?
    \item Welches Ergebnis hat die Arbeit?
    \item Eventueller Ausblick, falls die Arbeit nur ein Teil einer Gesamtstudie oder eines größeren Projekts ist.
\end{enumerate}


\section{Inhaltsverzeichnis}

\section{Abbildungs- und Tabellenverzeichnis}

\section{Einführung}

\begin{enumerate}
    \item Die Einführung beginnt mit einem Statement, das kurz die aktuelle Situation, in die die Arbeit eingebettet ist, umreißt:

„Aufgrund der starken Vernetzung von Produktionsanlagen werden Protokolle und Schnittstellen, die einen sicheren Datenaustausch gewährleisten, im Industrie-4.0-Umfeld immer bedeutsamer.“

„Bildgebenden Verfahren werden in immer stärkerem Maß zur Diagnose in der Medizin verwendet. Die Möglichkeit der stetig steigenden Auflösbarkeit der Daten erfordert jedoch auch neue Verfahren zu Verarbeitung der anfallenden Daten.“

„Aufgrund der zunehmenden Zahl an wissenschaftlichen Publikationen zur Validität von […] ist es wichtig, die einzelnen Studien übersichtlich zusammenzufassen und zu bewerten, was in diesem Bereich jedoch noch nicht systematisch durchgeführt wurde.“
    \item Im günstigsten Fall erkennt man hier schon eine generelle Forschungslücke. Wenn nicht, können sie jetzt die Forschungslücke allgemein beschreiben:

„Allerdings verwenden die etablierten Systeme […] keine End-to-End-Verschlüsselung, woraus die folgenden Probleme entstehen […]“

„Eine Schwachstelle bei der Verarbeitung derartiger Datenmengen stellt […] dar, da die bisherigen Verfahren […] nur unzureichend analysieren können. Eine Neu- oder Weiterentwicklung ist hier notwendig.“

„Aus diesem Grund ist es bisher nicht möglich, die bestehenden Verfahren neutral zu bewerten, Stärken und Schwachstellen gegenüberzustellen oder Empfehlungen für den Einsatz bestimmter Verfahren auszusprechen.“
    \item Da Sie nicht im luftleeren Raum arbeiten, stellen Sie im Folgenden schon vorhandene, grundlegende Forschungsergebnisse oder Erkenntnisse, auf denen Sie aufbauen wollen, überblicksartig dar.
Auf diese Weise präzisieren Sie die Forschungslücke und beschreiben, welche Teile der Lücke Ihre Arbeit füllen soll – und welche Erkenntnisse die Wissenschaft oder Technik daraus haben wird.
Dies ist noch nicht die genaue Beschäftigung mit der Theorie Ihrer Arbeit oder dem Stand der Technik, dies kommt erst im nächsten Abschnitt.
    \item 
Nachdem Sie den Rahmen, in dem Sie sich bewegen werden, abgesteckt haben und sich in Ihrem Bereich verortet haben, schließen Sie mit der Zielsetzung, die Sie mit Ihrer Arbeit verfolgen und leiten über in den nächsten Teil, die Darstellung der theoretischen Grundlagen Ihrer Arbeit bzw. dem Stand der Technik, auf dem Sie aufbauen
\end{enumerate}

\section{Theoretische Grundlagen}

\begin{itemize}
    \item Nehmen Sie sich Zeit für die Recherche, um nichts zu übersehen.
    \item Beschäftigen Sie sich intensiv mit der Literatur!
    \item Versuchen sie, Ihre Literatur zu strukturieren – in verschiedene (Teil-)Themenbereiche, nach Grundlagenarbeiten und Artikeln zur Anwendung, nach verschiedenen Lösungsansätzen oder nach der Relevanz für Ihre Aufgabe.
    \item Achten Sie darauf, dass Sie am Ende durch einen letzten Absatz überleiten in das nachfolgende Kapitel, die Beschreibung Ihrer Methodik. 
\end{itemize}

\section{verwendete Methoden und Werkzeuge}

\begin{itemize}
    \item Es hat sich bei vielen Arbeiten bewährt, entweder zu Beginn dieses Kapitels oder als eigenes Zwischenkapitel zuvor die Forschungsfragen, die Sie beantworten wollen, nochmals auszuformulieren und bei Bedarf auch noch in mehrere Detailfragen zu unterteilen.
    \item Im Anschluss beschreiben Sie, auf welche Weise die eingesetzten Methoden oder Verfahren dabei helfen, die einzelnen Fragen zu beantworten. Auf diese Weise wird Ihre Argumentation noch schlüssiger und auch Ihr Roter Faden ist noch besser zu erkennen.
    \item Führen Sie beispielsweise vor der Entwicklung eines innovativen UI Designs eine Befragung von potenziellen Usern durch, müssen Sie die Befragungsinstrumente, die geplante Methodik zur Auswertung und auch die Gewinnung der Probanden in diesem Abschnitt beschreiben.
    \item Handelt es sich um ausführliche Fragebögen, werden diese in den Anhang gepackt und im Text selbst lediglich zusammenfassend beschrieben. Die Grundlagen, wonach Sie die Fragebögen entwickelt haben, müssen im Theorieteil zu finden sein – wenn Sie sich auf Ihren Theorieteil explizit beziehen, stärken Sie nochmals Ihre Argumentation und festigen den Roten Faden.
    \item Haben Sie auf Basis einer psychologischen Theorie, die Sie auch in Ihrem Theorieteil literaturbasiert eingeführt haben, einen Fragebogen für die Probanden Ihrer UI-Design-Studie entwickelt, müssen Sie diesen Fragebogen theoriebasiert erläutern. Auch wie Sie Ihre Probanden gewinnen, wie Sie sicherstellen, dass deren Auswahl repräsentativ für die Gesamtgruppe ist oder ob Sie die Probandinnen und Probanden vielleicht sogar belohnen, gehört zum methodischen Teil. Schließlich sagen Sie damit auch etwas darüber aus, inwieweit man Ihre Ergebnisse verallgemeinern können wird, oder ob weitere Untersuchungen notwendig werden könnten.
    \item Sie müssen Ihre Methoden, Werkzeuge und Prozesse so genau beschreiben, dass grundsätzlich im Fach kundige Leserinnen und Leser Ihre Entscheidungen nachvollziehen und Ihre Arbeit danach im Wesentlichen mit ähnlichen Ergebnissen replizieren könnten.
\end{itemize}

\section{Durchführung des Forschungsvorhabens}

in diesem Kapitel in der Regel keine (oder zumindest kaum) Zitationen

\begin{itemize}
    \item Ziel einer konstruktiven Arbeit ist es, ein neuartiges Artefakt zu entwickeln, das bisher offene Probleme lösen kann. 
Im Kapitel zur Durchführung beschreiben Sie Ihr Vorgehen so, dass ein verständiger Dritter mit diesen Informationen Ihren Weg nachvollziehen oder replizieren kann. Dabei beziehen Sie sich immer wieder auf Ihre Forschungsfragen bzw. wie / warum Sie mit dem aktuellen Lösungsschritt eine der Fragestellungen beantworten können. 
Ergebnis am Ende der Realisierung ist ein fertiges Artefakt, das die anfangs definierten Zielvorgaben erfüllt und durch das die formulierten Forschungsfragen beantwortet werden. Die Zielerreichung sollte durch prototypische Erprobungen oder die Durchführung vorab definierter Tests dokumentiert werden. 
    \item Ziel einer analytischen Arbeit ist es, bestehende Ideen, Konzepte, Verfahren oder Artefakte zu analysieren, zu vergleichen oder zu bewerten, um damit neue Erkenntnisse in einem abgegrenzten Fach- oder Wissenschaftsbereich zu erhalten.
Im Kapitel zur Durchführung beschreiben Sie die Anwendung der im vorherigen Kapitel gewählten Methoden, die dabei auftretenden Schwierigkeiten oder nachträglichen Modifikationen.
Das Ergebnis dieses Kapitels ist die durchgeführte Analyse, die jedoch noch nicht zusammengefasst oder interpretiert wurde.
    \item Ziel einer theoretischen Arbeit ist es, eine neue Theorie oder Hypothese aufzustellen oder eine bestehende Theorie zu überprüfen, um damit neue Erkenntnisse für die Wissenschaft zu gewinnen. 
Das Kapitel zur Durchführung besteht aus Ihrer Beweisführung oder Hypothesenentwicklung, die Sie auch textuell erläutern. 
Das Ergebnis dieses Kapitels stellt der vollständig dokumentierte Beweis oder die abgeleitete Hypothese dar. 
\end{itemize}

\textbf{Gerade in den ersten Schreibphasen fällt es oft schwer, die Beschreibung der Durchführung von der Darstellung der Ergebnisse und der Diskussion der Ergebnisse zu trennen. Es ist nicht schlimm, wenn Ihnen das in der allerersten Version Ihrer Arbeit passiert - aber spätestens in der ersten Überarbeitung sollten Sie darauf achten, diese drei Kernabschnitte sauber voneinander zu trennen.}
\\

Vergessen Sie auch nicht, in jedem Fall das Kapitel mit einer Überleitung zur Darstellung der Ergebnisse zu schließen.

\section{Darstellung der Ergebnisse}

\begin{itemize}
    \item Haben Sie ein Softwareartefakt oder ein Modell entwickelt, stellen Sie hier die Funktion der Software oder Struktur des Modells dar, ergänzt durch Graphiken, Tabellen oder Testergebnisse. Strukturieren Sie die Beschreibung, so dass sie klar und eindeutig für den Leser bzw. die Leserin ist.
    \item Haben Sie eine Analyse durchgeführt, fassen Sie hier die Analyseergebnisse hier zusammen, ohne jedoch Erkenntnisse wegzulassen. Sie strukturieren die Ergebnisse so, dass sie zur Beantwortung der Forschungsfragen verwendet werden können, und ergänzen die Ergebnisse bei Bedarf durch Tabellen oder Graphiken.
    \item Bestand Ihre Arbeit aus einer Analyse, stellen Sie hier strukturiert Ihre Ergebnisse dar. Als Strukturierungshilfe können Sie Ihre Forschungsfragen verwenden, die ja durch Ihre Ergebnisse mehr oder weniger vollständig beantwortet werden. Alternativ können Sie auch eine thematische Gliederung verwenden. 
\end{itemize}

Um den Leser bzw. die Leserin in die Darstellung der Ergebnisse einzuführen, beziehen Sie sich zunächst auf schon zuvor bekannte Ergebnisse aus Ihrer Theoriearbeit oder auch auf Ihre Forschungsfragen:

"Eine immer wiederkehrende Frage zu Akzeptanzfaktoren von Softwareprodukten ist  [...], wie schon in Abschnitt x.x erläutert wurde. Die Befragung der Probandinnen und Probanden mittels des in v.y beschriebenen Fragebogens ergab, dass 24 Personen (n = 43 %) ...."
\\

Würden Sie feststellen, dass es für Ihre Ergebnisse in Ihrem bisherigen Text keine Anknüpfungspunkte gibt, hieße das, dass Sie entweder etwas in Ihrer Theoriearbeit vergessen haben, oder während der Realisierung von Ihrem geplanten Weg abgekommen sind.
\\

Probleme benennen!
\\

Wichtig: Um in Ihrer Arbeit die wissenschaftliche Ehrlichkeit zu wahren, ergänzen Sie im Anhang ALLE Ergebnisse - egal, ob Sie sie explizit vorgestellt haben oder nicht.

\section{Diskussion der Ergebnisse}

\begin{enumerate}
    \item Sie fassen sehr knapp Ihre Arbeit und Ihre Ergebnisse zusammen.
Sie stellen kurz Ihren konkreten Wissenschaftsbereich samt erkannter Forschungslücke dar (1 bis 2 Sätze). 
Sie fassen Ihr Forschungsprojekt zusammen und stellen dabei dar, was Ihre Arbeit von bestehenden Arbeiten unterscheidet (2 bis 3 Sätze)
    \item Sie ordnen Ihre Ergebnisse in die bestehende wissenschaftliche Theorie oder den Stand der Technik ein.
Dazu beziehen Sie sich auf Ihre Ausführung im Kapitel der Theoretischen Grundlagen, in dem Sie ja Ihren wissenschaftlichen Rahmen aufgespannt haben. Sollten Ihre Ergebnisse bisherigen Erkenntnissen widersprechen, diskutieren Sie mögliche Ursachen theoriegeleitet.
Unter Umständen zeigt sich in so einem Fall auch weiterer Forschungsbedarf (ein Punkt, auf den Sie im Ausblick verweisen können), oder Ihre Ergebnisse sind nicht allgemeingültig anzuwenden (die Ursache könnte in Ihrer Methodenwahl liegen) - auch dies müssen Sie diskutieren. 
    \item Sie zeigen, dass Sie Ihre gesetzten Ziele erreicht, die Forschungsfragen beantwortet haben.
Bei einer theoretischen Arbeit validieren Sie Ihre Ergebnisse und diskutieren die Auswirkungen und Nutzen für das Wissenschaftsgebiet.
Auch bei einer analytischen Arbeit validieren Sie Ihre Ergebnisse und interpretieren sie theoriegeleitet auf Basis Ihrer Vorarbeiten im Theorie- und Methodenteil.
Haben Sie im Rahmen einer konstruktiven Arbeit ein Artefakt entwickelt, können Sie die Zielerreichung durch einen Soll-Ist-Vergleich überprüfen, Sie können das Artefakt durch einen prototypischen Einsatz evaluieren oder Sie können das Ergebnis kriteriengeleitet validieren. 
    \item Sollten aufgetretene Probleme oder weitere Ereignisse einen Einfluss auf die Qualität des Projektergebnisses haben, wird dieser Einfluss hier ebenfalls diskutiert: Ergeben sich Limitationen hinsichtlich der Aussagekraft der Ergebnisse? Leitet sich vielleicht weiterer Forschungsbedarf ab, z.B. mit veränderten Rahmenbedingungen?
    \item Sie diskutieren den Beitrag Ihrer Arbeit zum wissenschaftlichen Erkenntnisgewinn.
\end{enumerate}

\section{Zusammenfassung und Ausblick}

Fast haben Sie es geschafft. Nun noch eine Zusammenfassung der wichtigsten Aspekte Ihrer Arbeit auf ca. einer Seite schreiben - und den Ausblick nicht vergessen. 
Zunächst: Je nach Arbeit, nach Thema oder auch nach verfübarem Raum (Fachartikel mit begrenzter Länge oder Masterarbeit mit wesentlich mehr verfügbarem Raum?) werden die beiden Teile zusammengefasst oder als einzelne Abschnitte geschrieben. Wie Sie es handhaben wollen, besprechen Sie mit Ihrer Betreuerin bzw. Ihrem Betreuer. 

\begin{itemize}
    \item Welchen weiteren Forschungsbedarf gibt es - sei es, dass Ihre Arbeit nicht alle Fragen beantworten konnten oder auch sich neue Fragestellungen durch Ihre Arbeit ergaben.
    \item Wie kann man - aufbauend auf den Erkenntnissen Ihrer Arbeit - weiter an der Fragestellung arbeiten, zum Beispiel durch Erweiterung des von Ihnen entwickelten Modells, durch Anwendung Ihrer Theorie oder durch Beschreibung von erweiterten Verfahren zu Ihrer Analyse.
    \item Stellt Ihre Arbeit einen Baustein einer größeren Forschungsarbeit dar, kann die Einbettung in das übergeordnete Forschungsprojekt und der Beitrag für dieses Gesamtprojekt dargestellt werden. 
\end{itemize}

\section{Ausblick (evtl. Trennen)}

\section{Literaturverzeichnis}

\section{Anhang}

\end{document}
